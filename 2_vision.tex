\section{Vision}
\begin{myremark}Technical visions slowly become true because of cheaper hardware, smaller hardware, wireless communication at almost no cost and sensors that provide context to objects.
\end{myremark}
\begin{center}
\begin{tikzpicture}[scale=0.9]
\begin{axis}[
x=1.5cm,
y=1.5cm,
axis y line=left,
axis x line=bottom,
xmax=8,xmin=0,
ymin=0,ymax=3,
xlabel=time,
x label style={at={(axis description cs:1,-0.1)},anchor=east},
ylabel=,
xmajorticks=false,
ymajorticks=false,
width=15cm,
anchor=center,
legend pos=south east,
legend cell align={right}
]
\addplot+ [samples = 40, domain=0:7.5, no markers, secondAccent, dotted] {x^3/130};
\addlegendentry{Number}
\addplot+ [samples = 40, domain=0:7.5, no markers, firstAccent] {-(x-7.5)^3/130};
\addlegendentry{Size}
\node at (1.25, 3) [anchor=north, align=center] {1 Computer \\ - Many People};
\addplot+ [mark=none, black, dashed] coordinates {(2.5, 0) (2.5, 3)};
\node at (3.75, 3) [anchor=north, align=center] {1 Computer \\ - 1 Person};
\addplot+ [mark=none, black, dashed] coordinates {(5, 0) (5, 3)};
\node at (6.25, 3) [anchor=north, align=center] {Many Computers \\ - 1 Person};

\end{axis}
\end{tikzpicture}
\captionof{figure}{Computing Trend}
\end{center}
\begin{mytitle}[Technological push and application pull] The technological push is complemented by an application pull because there is value for business, society and individuals.
\end{mytitle}
\begin{mytitle}[The evolution of networking]\hfill
\begin{itemize}
    \item Internet: network of computers, TCP/IP
    \item Web: network of documents, HTTP, HTML
    \item network of services, XML, WSDL
    \item network of smart things, JSON
\end{itemize}
\end{mytitle}
\begin{mytitle}[Smart objects] Smart objects are small, cheap and integrated sensors and processors with wireless communication.
\end{mytitle}
\newpage
\begin{mytitle}[Two important smart object paradigms]\hfill
\begin{itemize}
    \item Outsourcing smartness, so that the smart object doesn't need to provide everything itself as long as it can communicate with something that can do it instead.
    \item Outsourcing the user interface, so that the smart object can be made much cheaper.
\end{itemize}
\end{mytitle}
\begin{mytitle}[Bits vs. atoms] Bits implement all the smart behaviour, while atoms are the physical objects.
\end{mytitle}
\begin{mytitle}[Technological paternalism] Paternalism can be considered repressive, by protecting people and satisfying their needs but without allowing them any freedom or responsibility. Technological paternalism is already observed today, for example the beeps in a car if the seat belt is not fastened. The list of potential examples is growing.
\end{mytitle}
\begin{mytitle}[Reversal of defaults] Ron Rivest coined this term to describe the current trend where what once was private is now public, what once was hard to copy is now trivial to duplicate and what once was forgotten is now stored forever.
\end{mytitle}
