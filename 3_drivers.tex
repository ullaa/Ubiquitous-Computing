\section{Four Ubicomp Drivers}

\subsection{Moore's Law}
\begin{mytitle}[Moore's law] Processing speed and storage capacity doubles around every 18 months.
\end{mytitle}
\begin{mytitle}[Generalized Moore's law] The most important technology parameters like computation cycles, capacity of memory and bandwidth double every 1-3 years. But batteries and users mind-share do not. 
\end{mytitle}
\begin{mytitle}[Batteries] The efficiency of battery technology is improving only slowly over time. Considerations to take into account in respect to batteries include size, weight, cost, peak current vs. average current, rechargable vs. disposable, time to recharge, cycle life and ecological concerns. To save and gain energy we can enable power management in hardware units (shut down unused sections, reduce clock frequency and reduce voltage), use power-aware algorithms and harvest energy from the human body or the environment.
\end{mytitle}

\subsection{New Materials}
\begin{mytitle}[New materials] Examples of new materials are organic semiconductors, flexible displays and graphene. 
\end{mytitle}

\subsection{Progress in Communication Technologies}
\begin{mytitle}[Progress in communication technologies] Examples of new technologies are fiber optics, wireless 5G and bluetooth.
\end{mytitle}
\begin{mytitle}[Near-field communication (NFC)] NFC is short-range ($\sim$10 cm) interaction with handheld devices for example contactless payment. There are many advantages like using almost no energy, requiring only small transmitters and receivers, cheapness, security and no addressing or routing needed.
\end{mytitle}
\begin{mytitle}[Intrabody communication] Intrabody communication works by sending low-power electrical signals through the human body. This allows wearable devices to communicate and enables touch-selective communication. Some applications are that a car can recognize the driver, a shared device configures itself when being touched, devices identifying users and granting access and micro payments. Issues are the fear of phone radiation, safety concerns, reliability and security.
\end{mytitle}

\subsection{Better Sensors}
\begin{mytitle}[Better sensors] Examples of better sensors are miniaturized cameras and microphones, biometric sensors, temperature and humidity sensors, acceleration sensors and location sensors.
\end{mytitle}
\begin{myremark} Sensors are the interface between the real world and the cyber space.
\end{myremark}
\begin{mytitle}[Piezoelectric Effect] The piezoelectric effect describes the generation of energy resulting from an applied mechanical force. 
\end{mytitle}

\begin{mytitle}[Surface acoustic wave (SAW) based sensors] SAW based sensors need no battery or external power supply, they are powered by external RF interrogation signals or alternatively by physical actuation processes. They reflect the RF signal transmitted by an antenna up to 50 m away. The surface wave is a mechanical wave that propagates on the surface of a body, on piezo crystals for example it propagates at around 3500 m/s.
    \begin{mysubtitle}[Piezoelectric Transponder] The combination of a transducer and multiple reflectors is called a transponder, from transmitter-responder.
    \end{mysubtitle}
    \begin{mysubtitle}[How it works] When a signal arrives at the sensor, the transducer converts the electric energy from the RF waves to the surface acoustic wave. Each reflector then sends back parts of the wave thus encoding the response. The wave is then transformed back into an RF pulse sequence by the transducer and sent out. The surface wave is much slower than the RF wave, so a response takes more than 2 $\mu$s. This has the nice side effect that RF noise, e.g. reflections by the environment, is not mistaken as the answer by the antenna, because it decays after around 1 $\mu$s, so before the answer can arrive.
    \end{mysubtitle}
    \begin{mysubtitle}[Other SAW based sensors] One can add a second transformer at the end which changes the impedance. It is controlled by a resistor that depends on the sensor value.
    \end{mysubtitle}
\end{mytitle}
\begin{mytitle}[Applications]
    \begin{mysubtitle}[Applications for SAW] Some applications of SAW are temperature sensors, gas sensors, biosensors and tire sensors.
    \end{mysubtitle}
    \begin{mysubtitle}[Applications for battery-free sensors with transponders] Some applications are identification, temperature sensing, pressure sensing, transport monitoring, product tracking and material flow.
    \end{mysubtitle}
\end{mytitle}
\newpage
