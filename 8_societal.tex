\section{Societal Implications}
\begin{mytitle}[Nudging] Nudging is a mild and more accepted version of paternalism.
\end{mytitle}
\begin{mytitle}[Relevant technological development]\hfill
\begin{itemize}
    \item Computer vision: recognizing objects, situations, people, etc.
    \item Artificial intelligence: smart/autonomous systems, self-driving cars, etc.
    \item Wearable computing: components are distributed over the body and in clothes. They acquire context data and augment the users view of the environment. They have a hands free, intuitive user interface.
    \item Augmented reality: providing context information, intuitive interfaces, etc.
\end{itemize}
\end{mytitle}
\begin{mytitle}[Hybrid products] Hybrid products are physical items together with an added value provided by embedded ICT. They provide the user with a wealth of background information. The device is sold together with a service which makes it more difficult to imitate.
\end{mytitle}
\begin{mytitle}[New business opportunities] Tasks that could not be monitored can now be measured, controlled, managed and priced using embedded sensors and wireless feedback. We can measure the usage of a product or a service and then pay per use. Possible effects are more efficient markets, fairer prices, more adequate supply and better utilization of resources but also more stress for the consumer.
\end{mytitle}
\begin{mytitle}[Social risks] Social risks include smart things that may behave unexpectedly, self-determination, increased dependability and privacy concerns.
\end{mytitle}