\section{Economical Aspects}
\begin{mytitle}[Servitization] Servitization defines the departure from just producing physical goods towards a combination of physical goods and services. The desired continued economic sutainability (market pull) and technological innovation (technology push) drive servitization through digitalization. Because there is a large ``installed base'' and there are usually higher margins in the service business, there is an economic incentive to shift to aftermarket services.
\end{mytitle}
\begin{mytitle}[Risk and drawbacks]
    \begin{mysubtitle}[Risks for service providers] Risks include erosion of economics of scale, availability is king thus penalty clauses directly impact profitability, the temptation to be too ambitious with promises and that the promises to the customers might require additional resources.
    \end{mysubtitle}
    \begin{mysubtitle}[Risks for customers] Risks include the product/ecosystem lock-in and no legal ownership of the product means there are constraints on the own operations.
    \end{mysubtitle}
\end{mytitle}
\renewcommand{\arraystretch}{2}
\begin{center}
\begin{tabu} to 0.9\textwidth {  l | X[l] | X[l]  }
& Technology push & Market pull \\
\hline
Sense & New sensor technologies & Necessary to understand own products performance and cost\\
Connect & New communication technologies & Necessary to ship this data to back end\\
Analyze & New methods to acquire and process large amounts of data & Necessary to react fast to malfunctions or better forecast them\\
Control & New materials and softwareization & Even better to upgrade products remotely\\
Integrate & New architecures/infrastructures to integrate services & Products and services need to be flexible and future-proof
\end{tabu}
\captionof{table}{Technological push and market pull}
\end{center}
\begin{mytitle}[High-resolution management] The goal is to support management tasks like planning, leadership and controlling with automated data collection. This also means that digital business model patterns are becoming relevant to physical industries for the first time. 
\end{mytitle}
\begin{myremark} Traditionally information is not defined and is expected to be free of charge. The IoT business model sees information as a major source for value in the IoT.
\end{myremark}
\begin{mytitle}[New business model patterns]\hfill
\begin{itemize}
    \item Freemium: A product or service (typically a digital offering or an application such as software, media, games or web services) is provided free of charge, but money is charged for additional features, services, or virtual or physical goods. Examples are LinkedIn, Tinder and Dropbox.
    \item Digital add-on: Sell asset very inexpensively but customers can purchase higher-margin digital services. Invite third parties to sell add-ons as well. Examples include hardware add-ons like RAM sticks and software add-ons like additional language support.
    \item Digital lock-in: Limit the product compatibility to prevent counterfeits and ensure warranty. Customers are locked into a vendor’s world of products and services. Switching to another vendor is not possible without exposing yourself to substantial additional costs. Hence, this strategy protects the company from losing customers to competitors. Examples are Kindle, Gillette and Canon.
    \item Product as point-of-sales: The product itself takes on a marketing/sales role. An example is a smartphone on which one can then buy apps through the app store.
    \item Object self-service: The products are smart items that place orders on the internet themselves. One example is Amazons Alexa.
\end{itemize}
\end{mytitle}